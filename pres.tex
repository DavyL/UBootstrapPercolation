
% !TEX TS-program = pdflatex
% !TEX encoding = UTF-8 Unicode

% This file is a template using the "beamer" package to create slides for a talk or presentation
% - Giving a talk on some subject.
% - The talk is between 15min and 45min long.
% - Style is ornate.

% MODIFIED by Jonathan Kew, 2008-07-06
% The header comments and encoding in this file were modified for inclusion with TeXworks.
% The content is otherwise unchanged from the original distributed with the beamer package.

\documentclass{beamer}
\newtheorem{proposition}{Proposition}


% Copyright 2004 by Till Tantau <tantau@users.sourceforge.net>.
%
% In principle, this file can be redistributed and/or modified under
% the terms of the GNU Public License, version 2.
%
% However, this file is supposed to be a template to be modified
% for your own needs. For this reason, if you use this file as a
% template and not specifically distribute it as part of a another
% package/program, I grant the extra permission to freely copy and
% modify this file as you see fit and even to delete this copyright
% notice. 


\mode<presentation>
{
  \usetheme{Warsaw}
  % or ...

  \setbeamercovered{transparent}
  % or whatever 
}


\usepackage[english]{babel}
% or whatever

\usepackage[utf8]{inputenc}
% or whatever

\usepackage{times}
\usepackage{amsmath}
\usepackage[T1]{fontenc}
% Or whatever. Note that the encoding and the font should match. If T1
% does not look nice, try deleting the line with the fontenc.


\title[title] % (optional, use only with long paper titles)
{$\mathcal{U}$-Bootstrap percolation}

\author[Davy, Gjorgjevski, Pak] % (optional, use only with lots of authors)
{Leo Davy \and Martin Gjorgjevski \and Alexandre Pak}
% - Use the \inst{?} command only if the authors have different
%   affiliation.

\institute[ENS Lyon] % (optional, but mostly needed)
{
  ENS Lyon \\
  M2 Advanced Mathematics}
% - Use the \inst command only if there are several affiliations.
% - Keep it tildeple, no one is interested in your street address.

\date[Short Occasion] % (optional)
{March 2022}

\defbeamertemplate*{footline}{shadow theme}
{%
  \leavevmode%
  \hbox{\begin{beamercolorbox}[wd=.5\paperwidth,ht=2.5ex,dp=1.125ex,leftskip=.3cm plus1fil,rightskip=.3cm]{author in head/foot}%
    \usebeamerfont{author in head/foot}\insertframenumber\,/\,\inserttotalframenumber\hfill\insertshortauthor
  \end{beamercolorbox}%
  \begin{beamercolorbox}[wd=.5\paperwidth,ht=2.5ex,dp=1.125ex,leftskip=.3cm,rightskip=.3cm plus1fil]{title in head/foot}%
    \usebeamerfont{title in head/foot}\insertshorttitle%
  \end{beamercolorbox}}%
  \vskip0pt%
}
\setbeamertemplate{headline}
{%
  \leavevmode%
  \begin{beamercolorbox}[wd=.5\paperwidth,ht=2.5ex,dp=1.125ex]{section in head/foot}%
    \hbox to .5\paperwidth{\hfil\insertsectionhead\hfil}
  \end{beamercolorbox}%
  \begin{beamercolorbox}[wd=.5\paperwidth,ht=2.5ex,dp=1.125ex]{subsection in head/foot}%
    \hbox to .5\paperwidth{\hfil\insertsubsectionhead\hfil}
  \end{beamercolorbox}%
}

\setbeamertemplate{navigation symbols}{} 
\subject{Talks}
\begin{document}

\maketitle
\begin{frame}{Outline of the presentation}
	\begin{enumerate}
		\item Introduction to $\mathcal{U}$-bootstrap and examples
		\item Percolation and universality classes 
		\item Stable directions and critical densities
		\item Behaviour around phase transition
		\item Return to examples
		\item Continuity and noise sensitivity
		\item KCM
		\item Conclusion and open questions
	\end{enumerate}
\end{frame}
\begin{section}{ Introduction to $\mathcal{U}$-bootstrap and examples}
	\begin{frame}
		Definition and example of $r$-neighbour bootstrap
	\end{frame}
	\begin{frame}
		Definition for general $\mathcal{U}$\\
		Example for 1 or 2 rules families
	\end{frame}
	\begin{frame}
		Definition and simulation of $r$-neighbour
	\end{frame}
	\begin{frame}
		Definition and simulation of OP
	\end{frame}
	\begin{frame}
		Definition and simulation of spiral
	\end{frame}
	\begin{frame}
		Definition and simulation of DTBP
	\end{frame}
\end{section}
%%%%%%%%%%%%%%%%%%%%%%%%%%%%%%%%%%%
\begin{section}{Percolation and universality classes}
	\begin{frame}
		Motivation for the model \\
		Random initial set
	\end{frame}
	\begin{frame}
		Definition of $\theta_q(n)$ and $\theta_q$ \\
		Phase transition
	\end{frame}
	\begin{frame}
		Classification of local homogeneous monotone cellular automatas
	\end{frame}
	\begin{frame}
		Review of previous examples
	\end{frame}
\end{section}
%%%%%%%%%%%%%%%%%%%%%%%%%%%%%%%%%%%
\begin{section}{Stable directions and critical densities}
	\begin{frame}
		A geometric criterion to find the class of a model\\
		Stable/unstable directions\\
	\end{frame}
	\begin{frame}
		Universality classes based on stable directions
	\end{frame}
	\begin{frame}
		Application to $r$-neighbours (and explicit threshold for $r=2$)
	\end{frame}
	\begin{frame}
		Sketch of proof for universality classes
	\end{frame}
	\begin{frame}
		For supercritical and critical models, behaviour on $\mathbb{Z}^d$ is trivial. Need new tools for subcritical models\\
		Definition of $\tilde\theta $ and $  \tilde q_c $ .
	\end{frame}
	\begin{frame}
		Definition of critical densities $u\mapsto d_u$
	\end{frame}
	\begin{frame}
		\begin{theorem}{Theorem 3.1}
			$$\tilde q_c = \sup_u d_u$$
		\end{theorem}
	\end{frame}
	\begin{frame}
		(Sketch of ?) proof of theorem 3.1
	\end{frame}
	\begin{frame}
		Bounds on $\tilde q_c$
	\end{frame}
	\begin{frame}
		Use of OP to obtain bounds based on the trivial bound for DTBP
	\end{frame}
	\begin{frame}
		Interest of th 3.1 in some situations
	\end{frame}
	\begin{frame}
		Application for spiral model
	\end{frame}
\end{section}
%%%%%%%%%%%%%%%%%%%%%%%%%%%%%%%%%%%
\begin{section}{Behaviour around phase transition}
	\begin{frame}
		\begin{theorem}{Theorem 3.5}
			$$\tilde q_c $$ is where exponential decay happens
		\end{theorem}
	\end{frame}
	\begin{frame}
		Proof of th 3.5
	\end{frame}

	\begin{frame}
		Answers to questions 12 and 14 of Bolobas
	\end{frame}
	\begin{frame}
		Noise sensitivity and continuity
		\begin{theorem}{Theorem 3.6}
			
		\end{theorem}
	\end{frame}
\end{section}
%%%%%%%%%%%%%%%%%%%%%%%%%%%%%%%%%%%
\begin{section}{KCM}
	\begin{frame}
		Definition of KCM
	\end{frame}
	\begin{frame}
		\begin{theorem}{Theorem 3.7}
			
		\end{theorem}
	\end{frame}
	\begin{frame}
		Proof of theorem 3.7
	\end{frame}
\end{section}
%%%%%%%%%%%%%%%%%%%%%%%%%%%%%%%%%%%
\begin{section}{Conclusion and open questions}
	\begin{frame}
		Methods introduced allow to
		\begin{enumerate}
			\item Recover and generalize previously known results
			\item Study in depth and generality subcritical models
			\item Characterize the phase transition using $\tilde q_c$
			\item Obtain bounds on phase transition for subcritical models
			\item Answer previously open questions
		\end{enumerate}
		But they do not (yet) permit to answer the following questions
		\begin{enumerate}
			\item $q_c = \tilde q_c$ ?
			\item continuity of $u\mapsto d_u$
			\item dynamics on the torus
		\end{enumerate}
	\end{frame}
\end{section}
\end{document}


